\usepackage{cmap}
\usepackage[T2A]{fontenc}
\usepackage[utf8]{inputenc}
\usepackage{tabularx}
\usepackage{threeparttable}
\usepackage[14pt]{extsizes}
\usepackage{caption}
    \captionsetup{labelsep=endash}
    \captionsetup[figure]{name={Рисунок}}
\usepackage{amsmath}
\usepackage{enumitem}
\usepackage{geometry}
    \geometry{left=30mm}
    \geometry{right=10mm}
    \geometry{top=20mm}
    \geometry{bottom=20mm}
    \setlength{\parindent}{1.25cm}

\def\labelitemi{---}
\setlist[itemize]{leftmargin=1.25cm, itemindent=0.65cm}
\setlist[enumerate]{
    wide=\parindent,        % весь список выравнивается по абзацному отступу
    labelwidth=!,           % ширина метки подстраивается
    itemindent=!,           % отступ текста после метки — автоматически
    listparindent=\parindent % при переносе — отступ как в обычном абзаце
}
\usepackage{titlesec}
\newlength{\twospaces}
\settowidth{\twospaces}{шш}

% \makeatletter
% \renewcommand{\l@chapter}{\@dottedtocline{0}{0em}{1.5em}}
% \renewcommand{\l@section}{\@dottedtocline{1}{1.5em}{2.3em}}
% \renewcommand{\l@subsection}{\@dottedtocline{2}{3.0em}{3.2em}}
% \makeatother
% \titlespacing*{\chapter}{\parindent}{-30pt}{8pt}
% \titlespacing*{\section}{\parindent}{*4}{*4}
% \titlespacing*{\subsection}{\parindent}{*4}{*4}

\usepackage{setspace}
\onehalfspacing

\lstset{ %
	inputencoding=utf8,
    extendedchars=true,
	language=caml,                 % выбор языка для подсветки (здесь это С)
	basicstyle=\small\ttfamily, % размер и начертание шрифта для подсветки кода
	numbers=none,               % где поставить нумерацию строк (слева\справа)
	numberstyle=\tiny,           % размер шрифта для номеров строк
	stepnumber=1,                   % размер шага между двумя номерами строк
%	numbersep=5pt,                % как далеко отстоят номера строк от подсвечиваемого кода
	showspaces=false,            % показывать или нет пробелы специальными отступами
	showstringspaces=false,      % показывать или нет пробелы в строках
	showtabs=false,             % показывать или нет табуляцию в строках
	frame=single,              % рисовать рамку вокруг кода
	tabsize=2,                 % размер табуляции по умолчанию равен 2 пробелам
	captionpos=t,              % позиция заголовка вверху [t] или внизу [b] 
	breaklines=true,           % автоматически переносить строки (да\нет)
	breakatwhitespace=false, % переносить строки только если есть пробел
	escapeinside={\#*}{*)},   % если нужно добавить комментарии в коде
	abovecaptionskip=-5pt
}

\frenchspacing
\usepackage{indentfirst}

\usepackage{titlesec}
% \titleformat{\chapter}{\LARGE\bfseries}{\thechapter}{20pt}{\LARGE\bfseries}
% \titleformat{\section}{\Large\bfseries}{\thesection}{20pt}{\Large\bfseries}

\usepackage{multirow}
\usepackage{listings}
\usepackage{xcolor}


\usepackage{pgfplots}
\usepackage{tikz}
\pgfplotsset{compat=1.18}
\usetikzlibrary{datavisualization}
\usetikzlibrary{datavisualization.formats.functions}

\usepackage{graphicx}

\captionsetup[table]{justification=raggedright, singlelinecheck=false, labelsep=endash}
\captionsetup[lstlisting]{justification=centering, singlelinecheck=true}

\usepackage{csvsimple}
\newcommand{\code}[1]{\texttt{#1}}
\usepackage{gensymb}

\usepackage{makecell}
\renewcommand{\theadfont}{\bfseries}