\chapter{Исследовательская часть}

\section{Технические характеристики ЭВМ}
Замеры выполнялись на следующей машине:
\begin{itemize}
    \item процессор 11th Gen Intel Core i5-1135G7 4.20 ГГц~\cite{processor};
    \item оперативная память 8 Гб;
    \item операционная система EndeavourOS~\cite{endeavouros} Linux 6.17.9-arch1-1 x86\_64.
\end{itemize}
Указанный процессор имеет 4 физических ядра и 8 логических~\cite{processor}.
Исследование проводилось на ноутбуке, подключённом в сеть, при включённом режиме производительности. При выполнении замеров были запущены лишь системные процессы.

\section{Алгоритм проведения исследования}
Исследование построено по следующему алгоритму.

В качестве варьируемых параметров выбраны:
\begin{itemize}
    \item Глубина L-системы (число итераций) --- $n \in \{2, 3, 4, 5, 6\}$;
    \item Разрешение финального изображения --- $r \in \{256\times144, 426\times240, 640\times360, 854\times480, 1280\times720, 1920\times1080, 2560\times1440\}$;
    \item Разрешение карты теней --- $s \in \{128 \times 128, 256 \times 256, 512\times512, 1024\times1024, 2048\times2048, 4096\times4096\}$.
\end{itemize}

Остальные параметры зафиксированы:
\begin{itemize}
    \item аксиома: \texttt{"F"};
    \item правило генерации: \texttt{"F[+FL][-FL][\&FL][\^{}FL]"};
    \item угол ветвления: $25^\circ$;
    \item длина шага: $1.2$;
    \item базовый радиус ствола: $0.25$;
    \item коэффициент затухания радиуса: $0.6$;
    \item минимальный радиус листа: $0.02$;
    \item сегменты окружности: $8$;
    \item источник света: направленный, направление $(-1, -2, -1)$;
    \item камера: орбитальная, фокус на центре сцены;
    \item тени: включены.
\end{itemize}

Все замеры времени выполнялись с помощью функции
\texttt{std::chrono::high\_resolution\_clock}~\cite{iso_cpp_2020} 
и усреднялись по 10 запускам.
\section{Зависимость времени генерации и рендеринга от глубины L-системы}

В таблице~\ref{tbl:l_system_iterations} представлены результаты замеров времени генерации геометрии, времени рендеринга, количества треугольников и числа листьев от количества итераций.
\begin{table}[H]
    \caption{Замеры времени характеристик сцены от количества итераций L-системы}
    \label{tbl:l_system_iterations}
    \begin{tabular}{|r|r|r|r|r|}
        \hline
        \bfseries Итерации &
        \thead{Время \\ генерации, мс} &
        \thead{Время \\ рендеринга, мс} &
        \thead{Треуголь- \\ники} &
        \bfseries Листья \\
        \hline
        2 & 0.407 & 171.014 & 1\,184 & 74 \\
        \hline
        3 & 1.841 & 217.264 & 5\,746 & 508 \\
        \hline
        4 & 8.146 & 304.548 & 25\,317 & 2\,763 \\
        \hline
        5 & 40.899 & 535.007 & 112\,864 & 13\,951 \\
        \hline
        6 & 206.739 & 1\,319.613 & 525\,921 & 70\,140 \\
        \hline
    \end{tabular}
\end{table}

На рисунке~\ref{fig:stage1_1} представлен график зависимости времени генерации дерева от количества итераций L-системы.

\begin{figure}[H]
    \centering
    \includegraphics[width=0.9\textwidth]{images/stage1_gen_time_plot.pdf}
    \caption{График зависимости времени генерации дерева от количества итераций L-системы}
    \label{fig:stage1_1}
\end{figure}

На рисунке~\ref{fig:stage1_2} представлен график зависимости времени рендеринга кадра от количества итераций L-системы.

\begin{figure}[H]
    \centering
    \includegraphics[width=0.9\textwidth]{images/stage1_render_time_plot.pdf}
    \caption{График зависимости времени рендеринга кадра от количества итераций L-системы}
    \label{fig:stage1_2}
\end{figure}

Время генерации геометрии растёт по степенной зависимости: при увеличении глубины с 5 до 6 оно возрастает более чем в 5 раз.

Время рендеринга также следует степенной зависимости, но растёт медленнее, чем время генерации. Начиная с n=5, темп роста времени рендеринга заметно увеличивается.

\section{Зависимость времени рендеринга при различных размерах изображения}

В таблице~\ref{tbl:resolution_render_time} представлены результаты замеров времени рендеринга от разрешения изображения. При этом число глубины L-системы фиксировано и равно 5.

\begin{table}[H]
    \caption{Результаты замеров времени рендеринга от разрешения изображения (при фиксированной глубине L-системы $n=5$)}
    \label{tbl:resolution_render_time}
    \begin{tabularx}{\textwidth}{|X|r|}
        \hline
        \bfseries Разрешение изображения & \bfseries Время рендеринга, мс \\
        \hline
        256$\times$144   & 188.138 \\
        \hline
        426$\times$240   & 210.094 \\
        \hline
        640$\times$360   & 240.258 \\
        \hline
        854$\times$480   & 277.380 \\
        \hline
        1280$\times$720  & 363.993 \\
        \hline
        1920$\times$1080 & 533.528 \\
        \hline
        2560$\times$1440 & 757.984 \\
        \hline
    \end{tabularx}
\end{table}

На рисунке~\ref{fig:stage2} представлен график зависимости времени рендеринга от разрешения изображения.

\begin{figure}[H]
    \centering
    \includegraphics[width=\textwidth]{images/stage2_resolution_plot.pdf}
    \caption{График зависимости времени рендеринга от разрешения изображения}
    \label{fig:stage2}
\end{figure}

Время рендеринга растёт нелинейно с увеличением разрешения изображения, что соответствует степенной зависимости.


\section{Зависимость времени рендеринга изображения при различных размеров теневой карты}
В таблице~\ref{tbl:shadow_render_time} представлены результаты замеров времени рендеринга от разрешения теневой карты. При этом размер изображения фиксирован и равен $1920 \times 1080$, а количество итераций L-системы равно 5.

\begin{table}[H]
    \caption{Результаты замеров времени рендеринга от разрешения изображения (при фиксированной глубине L-системы $n=5$)}
    \label{tbl:shadow_render_time}
    \begin{tabularx}{\textwidth}{|X|r|}
        \hline
        \bfseries Разрешение теневой карты & \bfseries Время рендеринга, мс \\
        \hline
        128  & 430.055 \\
        \hline
        256  & 434.065 \\
        \hline
        512  & 443.957 \\
        \hline
        1024 & 466.171 \\
        \hline
        2048 & 533.541 \\
        \hline
        4096 & 763.706 \\
        \hline
    \end{tabularx}
\end{table}

На рисунке~\ref{fig:stage2} представлен график зависимости времени рендеринга от разрешения теневой карты.

\begin{figure}[H]
    \centering
    \includegraphics[width=\textwidth]{images/stage3_shadowmap_plot.pdf}
    \caption{График зависимости времени рендеринга от разрешения теневой карты}
    \label{fig:stage2}
\end{figure}

До разрешения $1024\times1024$ влияние теневой карты на производительность минимально: рост времени менее 9\%. Однако начиная с $2048\times2048$ наблюдается резкое увеличение нагрузки --- переход к $4096\times4096$ замедляет рендеринг на 43\%. Это связано с квадратичным ростом объёма теневого буфера и, как следствие, количества операций сравнения глубины при затенении каждого пикселя.

\section{Зависимость времени рендеринга изображения от угла обзора сцены}

В таблице~\ref{tbl:yaw_render_time} представлены результаты замеров времени рендеринга в зависимости от угла поворота камеры. При этом разрешение изображения фиксировано ($1920 \times 1080$), количество итераций L-системы равно 5, а разрешение теневой карты — 2048.

\begin{table}[htbp]
\centering
\caption{Зависимость времени рендеринга от угла обзора камеры}
\label{tbl:yaw_render_time}
\begin{tabularx}{\textwidth}{|c|X|}
\hline
\textbf{Угол обзора, градусы} & \textbf{Время рендеринга, мс} \\
\hline
0   & 1288.324 \\
\hline
45  & 1282.183 \\
\hline
90  & 1332.171 \\
\hline
135 & 1313.475 \\
\hline
180 & 1238.220 \\
\hline
225 & 1289.903 \\
\hline
270 & 1276.557 \\
\hline
315 & 1226.312 \\
\hline
\end{tabularx}
\end{table}

На рисунке~\ref{fig:stage3} представлена столбчатая диаграмма, отражающая зависимости времени рендеринга изображения от угла обзора сцены.
\begin{figure}[H]
    \centering
    \includegraphics[width=\textwidth]{images/stage4_camera_angle_plot.pdf}
    \caption{Cтолбчатая диаграмма зависимости времени рендеринга от угла обзора сцены}
    \label{fig:stage3}
\end{figure}

По полученным таблице и гистограмме видно, что разница между минимальным и максимальным временем рендеринга составляет около 8.6\%

\section{Результаты исследования}
Анализ полученных данных показал, что все исследуемые зависимости (время генерации от глубины L-системы, время рендеринга от разрешения изображения и от разрешения карты теней) имеют степенной характер.  

Наиболее резкий рост наблюдается для времени генерации геометрии: при увеличении числа итераций с 5 до 6 оно возрастает более чем в 5 раз, что связано с экспоненциальным ростом количества геометрических примитивов.  

Время рендеринга также заметно увеличивается с ростом разрешения, особенно при переходе к высоким значениям $1920 \times 1080$ и выше для изображения, $2048\times2048$ и выше --- для теневой карты. Это объясняется увеличением объёма данных, обрабатываемых на этапах растеризации и обработки теней.

В отличие от глубины L-системы или разрешения, угол обзора не влияет на объём генерируемой геометрии, поэтому его вклад в общее время рендеринга остаётся второстепенным.

\section{Вывод}
В данной части были приведены технический характеристики устройства, введена методология исследования, в соответствии с которой были проведены замеры, и проведён анализ полученных значений. 