\begin{center}
    \textbf{ВВЕДЕНИЕ}
\end{center}
\addcontentsline{toc}{chapter}{ВВЕДЕНИЕ}

Создание реалистичных трёхмерных моделей растений является важной задачей в компьютерной графике, особенно для визуализации природных ландшафтов, архитектурных симуляций и игровых сред. Одним из эффективных подходов к процедурной генерации деревьев служат L-системы --- формальные грамматики, предложенные Аристидом Линденмейером для моделирования роста биологических структур. Благодаря рекурсивному характеру и способности описывать самоподобные формы, L-системы широко применяются для генерации как двумерных, так и трёхмерных растительных объектов.

Цель курсовой работы --- разработка программного обеспечения, с графическим пользовательским интерфейсом для визуализации трёхмерных деревьев на основе L-систем с учётом параметров освещения, правил L-системы, углов ветвления и количества итераций генерации.

Для достижения цели поставлены следующие задачи:

\begin{itemize}
    \item провести сравнительный анализ существующих методов генерации трёхмерных деревьев и сделать выводы о применении L-систем для решения поставленной задачи;
    \item спроектировать метод генерации трёхмерных деревьев на основе L-систем и описать его;
    \item выбрать средства для разработки ПО и реализовать систему визуализации;
    \item провести исследование работы программы.
\end{itemize}

\clearpage